\chapter{Discussion}
\label{part:conclusions}

% NOT OK

With all major parts of the creative system discussed and the external evaluation results analysed, a conclusion on the results is given.
From this, a reflection on the designed system is given and the added value of this paper is discussed.
The paper ends by mentioning interesting future work and an important closing remark.


%------------------------------------
\section{Conclusions on the external evaluation}
\label{sec:conclusion_external_results}

The received ratings corresponded nicely with the intend of the images.
Artefacts of the system received high creativity scores but low ratings for the realism and car measure as they do not contain all required car components.
This makes them creative artefacts of the system that are interesting but not particularly useful for the domain.
Luckily these artefacts didn't occur often and would likely occur even less if the GAN were trained for even more epochs.

The system's learned concepts found through the extended GANSpace tool were also recognized by the participants, further confirming the labelling given for their role in the image generation.
The manual selection as replacement of the proposed similarity rating measure also proved successful since the only comparison to an existing car model was the one displayed in figure \ref{fig:similarcar}.
Whilst these cars do share similarities, they differ greatly when taking into account the similarities between designs in the domain.
It is not unlikely these two cars could co-exist in Bentley's car range. 
Many comments and mediocre ratings of similarity suggest this recognition of existing car brand styling traits occurs often.
Remember, this behaviour is often desired in the domain and thus positive for the system.

It is also shown that a generated car design can score great for reality, creativity and the other measure all at once.
With the car design shown in figure \ref{fig:survey_realistic} being the participant's favourite.
This generated design is indeed very intriguing, and thanks to the relatively realistic background it is not hard to imagine it being a press photo of a newly announced model.

The system is also capable of generating designs that can function as an inspiration for car designers even if they aren't perfect.
This was the case with the image shown in figure \ref{fig:missingpiece}.
The concept of a convertible truck is novel and whilst the image is not a perfect design due to missing pieces, it clearly shows the new concept and can be used as inspiration for car designers.




%------------------------------------
\section{Reflection on the developed creative system}
\label{sec:reflection_system}

The designed system is defended to be a creative system by explaining the different components based on Ventura's CC description paper as well as putting it into terms of the CSF.
The issues that GANs have to be accepted in the CC field were tackled, albeit mostly manual and not well generalisable for now.
However, the manual process can be automated and build into the proposed pipeline, given that there is enough computational power available.
These added measures don't limit the GANs capabilities as is also shown by this paper's system.
All of the different components are well documented and results can easily be recreated.
Readers are invited to play with the system and optimize it even further.

It isn't hard to imagine an optimized version of the system trained on a specific set of car models can function as an inspiration source for car designers.
This makes this creative system one that has clear potential in the industry as an aid for a creative profession rather than to replace it.
This is much like the system created by \citet{creativecargan}, which was made to optimize a car designers pipeline.
Since StyleGAN2 works so well for different types of image generation, it is also not hard to imagine a similar system being in place in different domains.
This would only require small changes to the proposed pipeline from figure \ref{fig:system_pipeline}.


%------------------------------------
\section{Added value}
\label{sec:added_value}

This paper discussed the shortcomings of GANs in the CC field.
By doing so, it also mentioned solutions that can be put in place to make GANs viable as creative systems.
These solutions were demonstrated in the creative system of this paper that is capable of generating novel car designs.
An external evaluation tool has been custom made and is available under the GPL V3 license.
This tool can be helpful for the CC field and other image evaluation settings. 


%------------------------------------
\section{Interesting future work}
\label{sec:future_work}

Some interesting future work has already been mentioned.
A summary is given below.
\begin{itemize}
    \item Train a StyleGAN2 model on a set of cars from one specific brand. Analyse the outputs together with a domain expert of that brand to further analyse the viability of such a system as an aid for a car designer.
    \item Implement the automatic similarity rating measure, ideally as part of the GAN and thus the training process.
    \item Implement a comparable system for a different domain, such as the clothing design domain.
    \item Perform an even broader external evaluation, possibly with car designers themselves.
\end{itemize}

%------------------------------------
\section{Closing remark}
\label{sec:closing_remark}

This paper made use of component analysis to discuss the system's capability of learning concepts.
This approach is widely used in literature to make deep NNs explainable.
An example of a paper that uses the same ideology is the mentioned paper by \citet{invidualunitanalysis}.
However, this approach is not without risk and a famous quote by E. Yudkowsky is given as a closing remark.


\epigraph{By far, the greatest danger of Artificial Intelligence is that people conclude too early that they understand it.}
{\textit{Eliezer Yudkowsky \\ AI theorist and writer}}
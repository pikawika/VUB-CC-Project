\chapter*{Abstract}

% OK

This paper discusses the development and evaluation of a creative system capable of generating photorealistic novel car designs and modifying them.
This system makes use of a pre-trained StyleGAN2 model \citep{stylegan2} and a modified version of the GANSpace tool \citep{ganspace}.
The various components are discussed loosely based on the computational creativity (CC) system description paper by \citet{ventura}.
These components are also placed inside the creative systems framework (CSF) proposed by \citet{csf} to further clarify the creative aspects of the system.

This paper also aims to discuss the possibilities and shortcomings of generative adversarial networks (GANs) in the CC field.
A more philosophical discussion is held to show such systems can indeed be creative rather than just generative.
It is shown how conceptual space exploration tools such as the modified GANSpace tool can be used to combat the black-box problems with GANs.
The need for CC specific internal evaluation and possible solutions are also briefly touched upon.
The external evaluation performed aims to further defend the creativity of the made system and thus the viability of GANs as a creative system.
The used tool for external evaluation was custom build for this project and is made available free to use and open source.

This paper was made as a requirement of the Computational Creativity course taught at the VUB.
All source files for this project are available on GitHub \citep{github_project}.
It is noted that this report is written using a modified version of the VUB based \LaTeX{} template from \citet{latex_template}. 
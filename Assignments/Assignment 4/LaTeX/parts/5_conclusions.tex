\part{Conclusions}
\label{part:conclusions}
% wat geleerd en wat onduidelijk

%------------------------------------
\section{What has been achieved so far}
\label{sec:issues}

During this assignment, much more than just the evaluation strategy was thought of.
It has been decided what pre-trained GAN will be used and it is working on a fresh Ubuntu install specifically for this project.
the GANSpace tool for control over the GAN is also already installed and even extended after some initial setup difficulties.
A lot of documentation on the setup and use of this tool has also been made on the GitHub repository for this project \citep{github_project}. 
Many interesting images from the GAN and some interesting components found using GANSpace have also already been found and documented on GitHub.

An easy to use, custom made, evaluation tool was also already created.
The tool has already been tested to work online and besides for some small remaining questions, the evaluating process can start soon.
All of this combined with the knowledge from the previous assignments means the road to the final assignment is clear and do-able.


%------------------------------------
\section{Expected roadmap}
\label{sec:roadmap}

Due to the huge progress made in this assignment, the expected roadmap introduced in the previous assignment could be shortened but remains unchanged just in case.
\begin{itemize}
    \item 02/04 - 11/04: Development of the GAN.
    \item 11/04 - 15/04: Control over the GAN and collection of images/videos for evaluation.
    \item 15/04 - 29/04: Collection of evaluation data.
    \item 29/04 - 18/05: Writing of the research paper.
\end{itemize}
\part{Conclusions}
\label{part:conclusions}
% wat geleerd en wat onduidelijk

%------------------------------------
\section{Proposed vision}
\label{sec:proposal}

The performed literature study has given a deeper insight into the viability and creativity of the proposed system from the previous milestone.
Taking into account this extra knowledge the following is proposed to be implemented by the next milestone:
\begin{itemize}
    \item A StyleGAN based system.
    \begin{itemize}
        \item Preferably based on a faster StyleGAN2 variant due to limited computational power.
    \end{itemize}
    \item Making use of existing labelled datasets as a time-saving measure.
    \item Having a fallback to existing pre-trained systems if the computational overhead would be too big or results non-satisfactory. 
\end{itemize}

%------------------------------------
\section{Hurdles}
\label{sec:hurdles}

The expected hurdles remain unchanged from the previous milestone being mainly:
\begin{itemize}
    \item A lack of insight on the working of the system.
    \item The need for more computational power then is available.
    \item Failing to objectively deem the system creative.
\end{itemize}
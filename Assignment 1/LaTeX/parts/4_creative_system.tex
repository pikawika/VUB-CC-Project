\part{The creative system}
\label{part:creative_system}

%------------------------------------
\clearpage
\section{Viability}
\label{sec:viability}

% PSA merge en sources van werking en mercedes AI based? (not full - second assignment is sources)
It has already been discussed that this topic is gaining interest in the car industry, thus the idea of an actual AI designed production car is likely to become reality.
The Hack Rod project already teased this idea, it is seen as the world's first AI-generated car \citep{hackrod}. 
More recently, the Czinger 21C is a 3D printed hypercar that claims to have used AI in several steps of the designing process \citep{czinger}.
This would have enhanced performance and lowered the cost according to Czinger.

Whilst these are interesting, they aren't viable for this project due to limited resources.
This project would limit itself to a more simple and fun DCGAN based on existing technology such as StyleGAN to produce images of cars created by the system.
No attention will be paid to whether or not these cars may be possible to be produced and driven, only the visual aspect is of importance for this project.

%------------------------------------

\section{System generated creativity}
\label{sec:generated_creativity}

%How my computational system will create in this domain (full)
It is planned to make a simple and fun DCGAN, as already pointed out previously.
This system will generate images of \textit{non existing} cars that take design cues from the cars of the input data.
Ideally, the situations proposed in section \ref{sec:new_era} would be addressable with the system.

In future milestones, it will be important to pay attention that the system is describable in terms of the creative systems framework. 
Ritchie's criteria and Jourdanous’s four perspectives will also have to be addressed as seen during the lectures.


%------------------------------------

\section{Available data sources}
\label{sec:data_sources}

%Computational data sources for my creative domain (full)
One of the issues with DCGAN is the sheer amount of data that is needed.
The next assignment will give more insight into this but it is not uncommon to need 50 thousand images or more.
Luckily some tricks could aid in gathering this data:
\begin{itemize}
    \item Taking images from car listing sites (e.g. CarMax).
    \item Using existing data sets.
    \item Starting from an already trained CNN.
\end{itemize}

This is again something the literature study in the next milestone will hopefully give more insight into.
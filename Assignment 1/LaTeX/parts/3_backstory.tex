\part{The backstory}
\label{part:backstory}

%------------------------------------

\section{A new era of cars}
\label{sec:new_era}

%A description of my creative domain (full)
The car industry is only just over a century old and has already evolved from a motorized luxury carriage for the rich to a multi-billion euro industry for the masses.
In the last decade, the car industry has undergone major changes, with electrification and autonomous driving being the most prominent.
Artificial intelligence plays a key role in these changes to insure save autonomous driving and optimal battery usage.

But AI can do much more for the car industry, it finds its usage in crash test simulations \citep{crashtest}, automotive aerodynamics \citep{carearo} and more.
This raises the question if the industry uses so many computer-generated simulations and calculations for validating the design of cars, can't a computer generate a car design?
This is a task that is gaining interest by big brands, especially in Formula 1 and hypercar design.

I would like to create a simple and fun Deep Convolutional Generative Adversarial Network (DCGAN) that can generate images of \textit{new cars} based on cars it has learned from. 
A similar task has already been done by \citet{cnnnewcar} and \citet{cybertruckguess}. 
These make use of StyleGAN \citep{stylegan}. 
The second milestone will go into more detail regarding relevant literature.

To be more specific I would like to introduce a creative system that can generate a creative interpretation of what a car could look like when design elements from different car models are mixed. 
This could give interesting results in different situations, for example:
\begin{itemize}
    \item PSA and FCA have recently merged their powers, what could a car look like when taking design cues from Peugeot (PSA) and Fiat (FCA)?
    \item If feeding the AI with the oldest and newest variant of a certain car model, does it generate a model that looks like an existing intermediate model?
    \item Can humans that are interested in cars link a certain generated car to a specific brand? 
\end{itemize}



%------------------------------------

\section{The car guy in me}
\label{sec:carguy}

%My personal background knowledge of my creative domain (full)
I've always said that, if you can't find me behind a computer screen, you'll most likely find me behind the steering wheel.
I've had an interest in the car sector since I was born and my dad has been a Peugeot mechanic for his whole life.
This gives me access to a lot of people who should be able to distinguish design cues from different car brands which can come in handy when evaluating the system.
The fact that I'm genuinely interested in such research and keep myself up to date on how AI is used in the car manufacturing process will hopefully also aid in the development of this system.  

%------------------------------------
\clearpage
\section{Viability}
\label{sec:viability}

% PSA merge en sources van werking en mercedes AI based? (not full - second assignment is sources)
It has already been discussed that this topic is gaining interest in the car industry, thus the idea of an actual AI designed production is likely to become reality.
The Hack Rod project already teased this idea, it is seen as the world's first AI-generated car \citep{hackrod}. 
More recently, the Czinger 21C is a 3D printed hypercar that claims to have used AI in several steps of the designing process \citep{czinger}.
This would have enhanced performance and lowered the cost according to Czinger.

Whilst these are interesting, they aren't viable for this project due to limited resources.
This project would limit itself to a more simple and fun DCGAN based on existing technology such as StyleGAN to produce images of cars created by the system.
No attention will be paid to whether or not these cars may be possible to be produced and driven, only the visual aspect is of importance for this project.